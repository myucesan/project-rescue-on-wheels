\documentclass[12pt]{article}
\usepackage[T1]{fontenc}
\usepackage{listings}
\usepackage{tikz}
\usepackage{booktabs}
\usepackage{tabularx}
\usepackage{graphicx}
\usepackage{array}
\usepackage[a4paper, margin=1in]{geometry}
\usepackage{color}
\usepackage{tikz}
\usetikzlibrary{matrix}
\usepackage{titlesec}
\usepackage[most]{tcolorbox}
\usepackage{lipsum}
\usepackage{times}
\usepackage[hidelinks]{hyperref}
\hypersetup{
	colorlinks,
	citecolor=black,
	filecolor=black,
	linkcolor=black,
	urlcolor=black
}

\setcounter{secnumdepth}{4}
\setcounter{tocdepth}{4}

\titleformat{\paragraph}
{\normalfont\normalsize\bfseries}{\theparagraph}{1em}{}

\titlespacing*{\paragraph}
{0pt}{3.25ex plus 1ex minus .2ex}{1.5ex plus .2ex}

\colorlet{helpful}{lime!70}
\colorlet{harmful}{red!30}
\colorlet{internal}{yellow!20}
\colorlet{external}{cyan!30}
\colorlet{S}{helpful!50!internal}
\colorlet{W}{harmful!50!internal}
\colorlet{O}{helpful!50!external}
\colorlet{T}{harmful!50!external}

\definecolor{mygreen}{rgb}{0,0.6,0}
\definecolor{mygray}{rgb}{0.5,0.5,0.5}
\definecolor{mymauve}{rgb}{0.58,0,0.82}

\newcommand{\texta}{Helpful\par \tiny (to achieve the objective)}
\newcommand{\textb}{Harmful\par \tiny (to achieve the objective)}
\newcommand{\textcn}{\rotatebox[origin=c]{90}{\parbox[t]{3cm}{\centering Internal origin\\ \tiny (Personal attributes)\par}}}
\newcommand{\textdn}{\rotatebox[origin=c]{90}{\parbox[b]{3cm}{\centering External origin\\ \tiny (Environmental attributes)\par}}}
\newcommand{\texten}{\rotatebox[origin=c]{90}{\parbox[t]{3cm}{\centering Internal origin\\ \tiny (Group attributes)\par}}}


\newcommand{\texts}{strength 1\par strength 2}
\newcommand{\textw}{weakness 1\par weakness 2}
\newcommand{\texto}{opportunity 1\par opportunity 2}
\newcommand{\textt}{threat 1\par threat 2}

\tcbset{swotbox/.style={size=normal, boxrule=0pt,
		colback=#1, watermark text=#1, width=.5\linewidth-5mm},
	header/.style={size=small, boxrule=0pt, width=.5\linewidth-5mm, colback=#1, valign=center, halign=center},
	firstcol/.style={header=#1, width=1cm}
}

\lstset{
	language=Java,
	commentstyle=\color{mygreen},
	frame=single,
	keepspaces=true,
	columns=fullflexible,
	keywordstyle=\color{blue},
	numbers=left,
	numbersep=5pt,
	numberstyle=\tiny\color{mygray}
}	
\begin{document}
	\title{\huge Rescue on wheels process documentation}
	\author{\\Team 2\\ \\Damian de Hoog 500780277\\ Yoshio Schermer 500760587\\Mustafa Y\"{u}cesan 500769574\\ Mohamed El Hadiyen 500777214}
	\maketitle
	\newpage
	\tableofcontents
	\newpage
	\section{Introduction}
	This document is an exploration of the process of creating the Metabot. In this document you'll gain insight into how the cooperation within the team started and evolved, agreements we made regards our cooperation, priorities set before and during the development and finally any problems encountered during the project.
	\newpage
	\section{Cooperation agreement}
	This chapter contains our cooperation agreement, this is a document that was created at the start of the rescue on wheels project and details made agreements and agreed upon rules of behaviour.
	\subsection{Communication}
	Our main communication channel is "Whats-app", we have made a group conversation in which we will discuss all things relevant to the project. If necessary we can use the built-in functionalities on Trello to communicate specific information about tasks but all other communication should be contained to the whats-app group. If a team member isn't reachable through the group conversation we might call him to make sure the information has come across properly.
	\subsection{Tools}
	During the project we will use the following tools (the tool names are hyper-links):
	\begin{itemize}
		\item \href{https://trello.com/b/VLCdhQAQ/sprint-3}{Trello} - Our digital Scrum board which we use to keep track of progress made.
		\item \href{https://gitlab.fdmci.hva.nl/hadiyem/row}{Gitlab} - Remote repository for code so that we can work together on the same files and have version control.
		\item \href{https://drive.google.com/drive/folders/1IJEu7igVdK12MzVx3jkXZMvB5HUXFuYZ?usp=sharing}{Google Drive} - We use Google drive for storing documents before they are posted on the website or added to the documentation files. Also on here are useful links and text from team members for team members.
		\item \href{https://metabotsrow.wordpress.com/}{Wordpress} - Our Wordpress blog which contains reports of daily scrum stand-ups, coaching and sprint retrospectives.
		\item PyCharm - IDE for python
		\item Intelij - IDE for Java
		\item Raspberry Pi 3B - Platform for the entire project.
	\end{itemize}  
	\subsection{Roles}
	During the project we have the following roles: "Scrum-master", "Product owner", "Documentation manager" and "Team member". These roles have been divided like so:
	\begin{itemize}
		\item Damian: Scrum-master; Sprint 1 \& 2, Documentation manager.
		\item Mohammed: Team member.
		\item Mustafa: Scrum-master; Sprint 3 \& 4.
		\item Yoshio: Product owner.
	\end{itemize}
	Scrum-master takes care of the daily standup, retrospective and our Trello board.\\
	Product owner is our communicator with outside parties regarding the project.\\
	Documentation manager plans all the documentation and oversees the creation of said documentation so that the other team members can work on it in an agile way and the manager will make sure it all adds up in the end.
	\subsection{Tasks}
	During daily stand-ups we will divide the tasks and check to see if everyone is still working on something. With the degree of variation that this project brings with it, it becomes hard to assign all tasks or a group of tasks to individuals. Tasks will be divided in agreement with each other and the assignment can be tracked on our \href{https://trello.com/b/VLCdhQAQ/sprint-3}{Trello} board.
	\subsection{Expectations}
	We expect each member to work on their assigned tasks and try to finish them. If problems arise, the scrum-master should be notified and the problem can be discussed during a stand-up. When this is not upheld the following should happen:
	\begin{itemize}
		\item Tell the team member that he should uphold the expectations
		\item Discuss said failed expectations within the group
		\item Bring it up during a coach meeting
	\end{itemize}
	\subsection{Educational goals}
	\textbf{Damian: }\emph{At the end of the project I am able to create more complex systems and manage a team as a scrum-master.}\\
	\textbf{Mohamed: }\emph{At the end of the project I am more proficient in the use of hardware.}\\
	\textbf{Mustafa: }\emph{At the end of the project I am more proficient in the use of robotics and hardware.}\\
	\textbf{Yoshio: }\emph{At the end of the project I am capable of guiding the team as scrum-master and I am able to relate the product owner's wishes as delegated product owner.}
	\subsection{Daily stand-up}
	Every week we will hold a daily stand-up on monday and tuesday morning. Things discussed during these stand-ups will be posted on our \href{https://metabotsrow.wordpress.com/}{Wordpress} blog. The blog will be maintained by a different team member each week. The schedule for this can be found on our \href{https://drive.google.com/drive/folders/1IJEu7igVdK12MzVx3jkXZMvB5HUXFuYZ?usp=sharing}{Google Drive}. At the end of each project day we will hold another stand-up to review progress made and problems encountered during said project day.
	\subsection{Absence}
	In case of sickness or other absences, let the team know as soon as possible and try and finish your tasks at home. If as task is being done by multiple people, a re-assignment of the task might be needed and this will be brought up during the daily stand-up. In case of serious illnesses, the team member is released from his task(s) to ensure rest and optimal recovery.
	\newpage
	\section{SWOT Analysis}
	To start off our project in the best possible way, we needed to understand our group dynamic. How our group will behave and perform. To gain a better understanding of this dynamic our coach tasked us with the development of a 'Strengths, Weaknesses, Opportunities and Threats (SWOT) analysis'.\\ This chapter contains said analysis. First, SWOT analyses of each individual member which are then combined to create a group analysis. This approach was chosen because the group itself may create different characteristics but it still consists of individual members. If we understand the members as individuals, we may get a better understanding of the group. 
	
	\subsection{Individual analyses}
	\subsubsection{Damian}
	\begin{tcbitemize}[raster columns=3, raster rows=3, enhanced, sharp corners, raster equal height=rows, raster force size=false, raster column skip=0pt, raster row skip = 0pt]
		
		%Empty corner and two headers
		\tcbitem[blankest, width=1cm]
		\tcbitem[header = helpful]
		\texta
		\tcbitem[header = harmful]
		\textb
		
		%First row
		\tcbitem[firstcol = internal]
		\textcn
		\tcbitem[swotbox = S]
		\begin{itemize}
			\item I am a quick learner.
			\item I am a calm and collective individual.
			\item If I am interested in something i can spend all my time on it.
			\item I have a decent amount of wisdom.
			\item I am capable of quality self reflection.
			\item I am a curious individual.
			\item I yearn for self-improvement.
			\item I have great leadership and conflict resolution skills.
		\end{itemize}
		\tcbitem[swotbox = W]
		\begin{itemize}
			\item I tend to procrastinate and only switch to higher gear under pressure of deadlines.
			\item I find my technical skills lacking in comparison to my team members.
			\item If I'm not completely interested or intrigued by something, I find it hard to get myself to work for or on it
		\end{itemize}
		
		%Second row
		\tcbitem[firstcol = external]
		\textdn
		\tcbitem[swotbox=O]
		\begin{itemize}
			\item My project team has a lot of bright minds from which I might be able to learn technical skills and improve my current ones.
			\item During this project I have the opportunity to practice the Scrum-master role.
		\end{itemize}
		\tcbitem[swotbox=T]
		\begin{itemize}
			\item The rover has proven to be unreliable in the hardware department and might fail once again.
			\item My team, which has more technical skills than me, might outperform me.
		\end{itemize}
	\end{tcbitemize}
	\newpage
	\subsubsection{Mohamed}
	\begin{tcbitemize}[raster columns=3, raster rows=3, enhanced, sharp corners, raster equal height=rows, raster force size=false, raster column skip=0pt, raster row skip = 0pt]
	
		%Empty corner and two headers
		\tcbitem[blankest, width=1cm]
		\tcbitem[header = helpful]
		\texta
		\tcbitem[header = harmful]
		\textb
	
		%First row
		\tcbitem[firstcol = internal]
		\textcn
		\tcbitem[swotbox = S]
		\begin{itemize}
			\item When I work on a product, I am very determined and dedicated to finishing that product and maintain quality at the same time: I will not give up till it is completely done.
			\item  
			I like to help my colleagues if they get stuck with their task during a project
			\item I am capable of having a stable communication with my colleagues, because I am known to be a very social and kind person.
			\item I am totally open to sharing my knowledge with someone else. 		
		\end{itemize}
		\tcbitem[swotbox = W]
		\begin{itemize}
			\item I am known to be a perfectionist. I want to finish the product I am working on in the best way possible. However, most of the time I tend to waste lots of precious time because I am pondering over how I can work on the product the best way possible.
			\item I have a hard time handing out tasks during a project because I do not know whether my colleagues will be able to finish a certain task or not. 
			\item I have a hard time to wake up in the morning because I am definitely not a morning person.
		\end{itemize}
	
		%Second row
		\tcbitem[firstcol = external]
		\textdn
		\tcbitem[swotbox=O]
		\begin{itemize}
			\item I would like to meet new people to broaden my network in the future.\\
			\item I will be able to learn about new technologies on hardware.
		\end{itemize}
		\tcbitem[swotbox=T]
		\begin{itemize}
			\item As I only got to know most of the students in my team recently, there may be a student in my team who does not put much effort in the project.
			\item As I am a excellent student, I may get more pressure from my non-excellent students in the project, which could affect me badly.
		\end{itemize}
		\end{tcbitemize}
	\newpage
	\subsubsection{Mustafa}
	\begin{tcbitemize}[raster columns=3, raster rows=3, enhanced, sharp corners, raster equal height=rows, raster force size=false, raster column skip=0pt, raster row skip = 0pt]
		
		%Empty corner and two headers
		\tcbitem[blankest, width=1cm]
		\tcbitem[header = helpful]
		\texta
		\tcbitem[header = harmful]
		\textb
		
		%First row
		\tcbitem[firstcol = internal]
		\textcn
		\tcbitem[swotbox = S]
		\begin{itemize}
			\item Not running away from asking questions.
			\item Healthy curiosity.
			\item Not afraid of confrontation but usually can't be bothered with it.
			\item Doesn't mind when bossed around as long as it's leveled.
			\item Knows how to get along with people if he really wants to.
		\end{itemize}
		\tcbitem[swotbox = W]
		\begin{itemize}
			\item Quickly annoyed when someone has an improper way of speaking.
			\item Doesn't appreciate people being or talking authoritarian to him.
		\end{itemize}
		
		%Second row
		\tcbitem[firstcol = external]
		\textdn
		\tcbitem[swotbox=O]
		\begin{itemize}
			\item Can get hands on experience with robotics.
			\item Can expand on his hardware programming.
		\end{itemize}
		\tcbitem[swotbox=T]
		\begin{itemize}
			\item Sudden relationship which drags his life downhill.
			\item Lives more than an hours trip away from school while his colleagues are near.
			\item Started a week later with school which makes him 1 week late with his homework.
			\item 2-days a week work.
			\item Excellence subjects which are time consuming.
		\end{itemize}
	\end{tcbitemize}
	\newpage
	\subsubsection{Yoshio}
	\begin{tcbitemize}[raster columns=3, raster rows=3, enhanced, sharp corners, raster equal height=rows, raster force size=false, raster column skip=0pt, raster row skip = 0pt]
		
		%Empty corner and two headers
		\tcbitem[blankest, width=1cm]
		\tcbitem[header = helpful]
		\texta
		\tcbitem[header = harmful]
		\textb
		
		%First row
		\tcbitem[firstcol = internal]
		\textcn
		\tcbitem[swotbox = S]
		\begin{itemize}
			\item Able to balance perfectionism with productivity.
			\item Perfectionist.
			\item Productive.
			\item Empathetic.
		\end{itemize}
		\tcbitem[swotbox = W]
		\begin{itemize}
			\item Appears to look irritated even when not, causing some people being afraid to ask/comment something.
			\item Difficult to express appreciation even when truly happy with some result.
		\end{itemize}
		
		%Second row
		\tcbitem[firstcol = external]
		\textdn
		\tcbitem[swotbox=O]
		\begin{itemize}
			\item With the role, delegated product owner, I am able to get more experience in negotiation and communication.
			\item With the role, delegated product owner, I am able to have more control over what we can make.
		\end{itemize}
		\tcbitem[swotbox=T]
		\begin{itemize}
			\item Two excellence students in the team. They have a higher possibility of being arrogant. This could result in being pushed over by them.
			\item Two students in the team work 8 up to 20 hours every week.
		\end{itemize}
	\end{tcbitemize}
	\newpage
	\subsection{Group analysis}
	\begin{tcbitemize}[raster columns=3, raster rows=3, enhanced, sharp corners, raster equal height=rows, raster force size=false, raster column skip=0pt, raster row skip = 0pt]
		
		%Empty corner and two headers
		\tcbitem[blankest, width=1cm]
		\tcbitem[header = helpful]
		\texta
		\tcbitem[header = harmful]
		\textb
		
		%First row
		\tcbitem[firstcol = internal]
		\texten
		\tcbitem[swotbox = S]
		\begin{itemize}
			\item Social
			\item Helpful to one another
			\item No dictators
			\item Smart
			\item Curious
			\item Desire to self-improve
			\item Empathetic
			\item Desire to create a quality product
		\end{itemize}
		\tcbitem[swotbox = W]
		\begin{itemize}
			\item Perfectionist
			\item Low confidence in one another
			\item Difficulty with being on time
			\item Procrastination
			\item Different skill levels amongst the team
			\item Authority/Following commands problems
		\end{itemize}
		%Second row
		\tcbitem[firstcol = external]
		\textdn
		\tcbitem[swotbox=O]
		\begin{itemize}
			\item Social networking
			\item Learning new technologies
			\item Experiencing negotiation and communication
			\item Learn from team members
			\item Learn and improve scrum knowledge
			\item Experiencing a robotics project
		\end{itemize}
		\tcbitem[swotbox=T]
		\begin{itemize}
			\item Ignorance from teammates
			\item Increased pressure from excellence program
			\item Two team members work 10 to 20 hours a week
			\item Rover appears to be unreliable
			\item Fear of out-performance by other team members
			\item Differences in travel time
			\item Excellence program extra courses and time needed for them
		\end{itemize}
	\end{tcbitemize}
	
\end{document}	